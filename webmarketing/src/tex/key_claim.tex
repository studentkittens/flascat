\label{keyvisual_claim}
\chapter{Keyvisual und Claim}

Beim Key-Visual wurde als Logo ein Elch mit Saiten und Musiknoten im Geweih
gewählt.
\\
Die Wahl eines Tieres als Logo ist hier leicht an das Vorbild Napster
angelehnt welches eine Katze mit Kopfhörern zeigt.

\begin{center}
    \includegraphics[scale=0.4]{./gfx/napster.jpeg}
\end{center}

Es wurde bewusst ein Elch ausgewählt, dieser soll symbolisch die anfangs
überwiegend aus den skandinavischen Ländern antreibende Kraft der Indie Musik
Bewegung visuell darstellen. Desweiteren sollen die Saiten im Geweih den
Zusammenhalt der Communiy wiederspiegeln, der grimmige Blick ist eine
,,Kampferklärung'' an die Mainstream Musik Branche. Der comichafte Ansatz soll
hier primär ein junges Publikum ansprechen.

\section{Key-visual}
\begin{center}
    \includegraphics[scale=0.4]{./gfx/elchlogo.png}
\end{center}

\paragraph{Beschreibung} Das Logo besteht aus einem comichaft gezeichneten
Elchkopf, der zwischen den Lücken seines Geweihs Saiten aufgespannt hat.
Zwischen diesen sind 4 Musiknoten in unterschiedlicher Größe und Ausrichtung 
dargestellt. 
\\
\\
Neben dem Maskottchen ist der verwendete Schriftzug zu finden.
In großer Schrift (Font: Monoton 110) ist dabei der Name unser Dienstleistung
geschrieben, darunter in Schreibschrift (Font: Leckerli One 28) unser Claim.
\\
\\
Der grimmige Blick ist bewusst gewählt worden um uns von anderen
ewig-glücklichen Maskottchen abzugrenzen.


\section{Claim}

Der Claim soll unsere Dienstleistung möglichst prägnant zusammenfassen.
Das ,,bringt's'' bezieht sich dabei auf das Liefern von Neuigkeiten und
Metadaten verschiedener Art. Er ist möglichst kurz ausgewählt worden, 
und kann auch leicht ins Englische übersetzt werden (,,moosr delivers'').
\\
\\
Ein solch kurzer Claim kann auch auf Fanartikel wie T-Shirts oder kleine
Werbgeschenke wie Kugelschreiber gedruckt werden.
