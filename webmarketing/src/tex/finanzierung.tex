\chapter{Finanzierung}

\section{Lizenz}
Primär ist es uns wichtig den Bekanntheitsgrad zu steigern und eine große
Community zu gewinnen welche auch daran interessiert ist die Dienstleistung
weiter auszubauen und zu verbessern. Die direkte Kommunikation zu der Community
soll eine Weiterentwicklung der Dienstleistung gewährleisten und dabei
gleichzeitig die Interessen der ,,Kunden'' wahren.
\\
Aus diesem Grund haben wir uns für Freie Lizezen entschieden um so jedem die
Möglichkeit zu bieten die Dienstleistung zu verbessern.

Alle Quelltexte stehen unter GPLv3 Lizenz.
\url{http://www.gnu.org/licenses/gpl-3.0.html}
Das \emph{Design} sowie alle weiteren
Medien unter der Creative Commons Lizenz.
\url{http://de.creativecommons.org/ }

\section{Mögliche Finanzierungsszenarien}
Da unsere Dienstleistung nicht kommerziellen Ursprungs ist trifft diese Frage
nicht direkt zu.
\\
Als primäre Finanzierungsquelle um die Serverkosten zu decken bieten wir über
Drittanbieter wie 

\begin{itemize}
\item http://www.zazzle.de/
\item http://www.cafepress.de/
\end{itemize}

Fanartikel wie T-Shirts, Aufkleber etc. an. Durch diese Maßnahme wird zusätzlich
als positiver Nebeneffekt Werbung geschaffen.
\\
Mögliche Einnahmequellen wären Spenden und mögliche kostenpflichtige API
Anpassungen für ,,Kunden'' kommerzieller Musikplayer und Contentanbieter. Wir
erhoffen uns durch die freigewählten Lizenzen, dass kommerzielle Dienstleister
ermutigt werden unseren Dienst zu finanzieren und somit zur Entwicklung direkt
oder indirekt beizutragen.
\\
Der Dienst an sich ist prinzipiell selbsttragend, da außer den
Unterhaltungskosten für Domain und Server keinerlei laufende Kosten anfallen sollten.
