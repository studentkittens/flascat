\chapter{Konzept Web}

\section{Allgemein}

Unsere Dienstleistung konzentriert sich ähnlich der Unix Philosophie auf eine
bestimmte Tätigkeit und versucht diese möglichst gut zu realisieren. Aufgrund
dieser Tatsache ist es uns sehr wichtig den User nicht mit ,,Thematikfremden''
Themen zu nerven. Aufgrund unserer Spezialisierung werden fachfremde Kategorien
wie z.B. der Fanartikel Webshop an Drittanbieter ,,outgesourct''.
\\
\\
Um eine einfache sowie angenehme Usability zu gewährleisten konzentrieren wir
uns sehr stark darauf das \emph{Design} nach dem KISS-Prinzip so einfach wie
möglich zu halten und den Benutzer nicht mit Werbebannern oder sinnvollen
Informationen zu belästigen.

\newpage

\section{Farbklima}
Das an den Key-Visual angepasste Farbklima sieht wie folgt aus:

\begin{table}[h!]
\centering
\begin{tabular*}{\textwidth}{lll|l}

    Farbe & Name & Farbwerte & Einsatzgebiet \\
    \hline
    
\multirow{3}{*}
    {
    \begin{tikzpicture}
        \draw [line width=0.5pt,fill=a]
        (0,0) -- (1.2,0) -- (1.2,1.2) -- (0,1.2) -- cycle;
    \end{tikzpicture}
    }
    & a & HEX \#FDF9EF & Hintergrundfarbe für Contentbereich\\
    & & RGB 253, 249, 239 \\
    & &  \\
    \hline

\multirow{3}{*}
    {
    \begin{tikzpicture}
        \draw [line width=0.5pt, fill=b]
        (0,0) -- (1.2,0) -- (1.2,1.2) -- (0,1.2) -- cycle;
    \end{tikzpicture}
    }
    & b & HEX \#F4F1E5 &  Hintergrund \\ 
    & & RGB  244, 241, 229 & \\
    & &  \\
    \hline

\multirow{3}{*}
    {
    \begin{tikzpicture}
        \draw [line width=0.5pt,fill=c]
        (0,0) -- (1.2,0) -- (1.2,1.2) -- (0,1.2) -- cycle;
    \end{tikzpicture}
    }
    & c & HEX \#544738 &  Standard Schriftfarbe \\
    & & RGB 84, 71, 56 & \\
    & &  \\
    \hline

\multirow{3}{*}
    {
    \begin{tikzpicture}
        \draw [line width=0.5pt,fill=d]
        (0,0) -- (1.2,0) -- (1.2,1.2) -- (0,1.2) -- cycle;
    \end{tikzpicture}
    }
    & d & HEX \#98BF21 &   Buttonflächen \\
    & & RGB 152, 191, 33 & \\
    & &  \\
    \hline

\multirow{3}{*}
    {
    \begin{tikzpicture}
        \draw [line width=0.5pt,fill=e]
        (0,0) -- (1.2,0) -- (1.2,1.2) -- (0,1.2) -- cycle;
    \end{tikzpicture}
    }
    & e & HEX \#C2C5BE &  Footer Hintergrundfarbe \\
    & & RGB 194, 197, 190 &\\ 
    & &  \\
    \hline
   
\multirow{3}{*}
    {
    \begin{tikzpicture}
        \draw [line width=0.5pt,fill=f]
        (0,0) -- (1.2,0) -- (1.2,1.2) -- (0,1.2) -- cycle;
    \end{tikzpicture}
    }
    & f & HEX \#BBBBBB &  Tagcloud Schriftfarbe \\
    & & RGB 187, 187, 187  & \\
    & &  \\
    \hline
   
\multirow{3}{*}
    {
    \begin{tikzpicture}
        \draw [line width=0.5pt,fill=g]
        (0,0) -- (1.2,0) -- (1.2,1.2) -- (0,1.2) -- cycle;
    \end{tikzpicture}
    }
    & g & HEX \#303030 &  Tagcloud Farbe \\
    & & RGB 48, 48, 48  & \\
    & &  \\

\end{tabular*}
   \caption{Farbschema}
   \label{t_colorscheme}
\end{table}

\section{Webservice Kategorien}
Folgende Hauptkategorien wurden auf moosr umgesetzt:

\paragraph{moosrdata}
Hier befindet sich die der Kernpunkt unserer Dienstleistung, die Metadaten
Suchmaschine. Diese Seite bietet dynamischen Content in Form einer
\emph{Tagcloud} mit den zu einem bestimmten Zeipunkt am häufigsten gesuchtenn
Metadaten.

\paragraph{Developers}
Die Kategorie \emph{Developers} ist eine Seite mit statischem Content. Hier
werden für interessierte Entwickler alle Informationen zu unserer HTML und JSON
API erläutert.

\paragraph{Blog}
Der \emph{Blog} ist eine Seite mit wachsendem Content. Über diese Seite wollen
wir unsere Community auf dem laufenden halten über Newcomer aus der Szene und
Berichte über Alben, Festivals etc. verfassen.  

\paragraph{Webshop}
In der Kategorie \emph{Webshop} befinden sich Seitenlinks und Beschreibungen zu
Webshops mit Tickets und Fanartikeln. Diese Seite enthält hauptsächlich
statischen Content.

\paragraph{Forum}
Das \emph{Forum} ist der Haupttreffpunkt für die Community. Hier können sich
alle nach Lust und Laune austauschen. Hier ist der Content auch statisch, wächst
jedoch mit der Anzahl der Besucher und Beiträge.

\paragraph{About Us}
In der Kategorie \emph{About Us} erläutern wir dem Kunden das Spektrum unserer
Dienstleistung. Diese Seite ist zugleich unsere \emph{Hauptseite}. Diese Seite
enthält wieder statischen Content der unsere Dienstleistung klar und direkt
beschereibt.

\paragraph{F.A.Q.}
Der \emph{F.A.Q.} Bereich bietet interessierten Antworten auf die häufig
gestellen Fragen. Der Content ist statisch und ändert sich kaum. 

\paragraph{Login}
Hier ist der \emph{Loginbereich} der primär für \emph{Redakteure} interessant
ist die Beiträge auf der Blogseite veröffentlichen möchten. Auf der Loginseite
selbst ist kein weiterer zusätzlicher Content vorhanden.
\\
\\
Jeglicher Content in allen Kategorien wird ,,per Hand'' verfasst. Kopieren von
Texten ist verboten. Dadurch wollen wir die Orginalität unseres Contents
gewährleisten.

\section{Footer-Bereich}
Im \emph{Footer} befinden sich Links zu folgenden Seiten:

\paragraph{Social}
Im \emph{Social} Bereich befinden sich Links auf die folgenden sozialen Netze in
welche unsere Dienstleistung Vorganden ist:
\beigin{itemize}
\item Flickr (Bildergallerien von Festivals und Community-Treffs)
\item Google+ (Unser Auftritt bei Google+)
\item Facebook (Unser Auftritt bei Facebook)
\item Twitter (Unser Microblogging Seite um kurz über Änderungen auf unserer
Seite zu informieren)
\item E-Mail Kontakt
\item RSS Feed zum Blog
\end{itemize}

Wir haben uns bewusst für die Auslagerung von großen Bildergallerien auf Flickr
entschieden weil wir so Bilderreihen von Festivals oder Community-Treffs online
stellen können ohne dass hierbei Datenvolumen Kosten auf unserer Seite
entstehen. Gleichzeitg sparen wir Speicherplatz auf unserem Server.
\\
\\
Die Einbindung von Flickr und den weiteren sozialen Netzen hat neben dem
,,Mundpropaganda Werbeeffekt'' den Vorteil dass hier gleichzeitig ein
,,Hotspot'' auf unsere Seite generiert wird was unser Google Ranking verbessert.

\paragraph{Sitemap}
Der Sitemap bereich bietet nochmal für den Google-Crawler die Möglichkeit in
alle Bereiche unserer Seite zu kommen.
\paragraph{Featured}
Hier sind Links zu den von uns verwendeten Libraries wie z.B. libglyr. Dadurch
informieren wir nicht nur über die Technologien die wir verwenden sondern
ermöglichen es der Community auch indirekt an moosr über die Entwicklung an
Thirdpartielibraries mitzuarbeiten.
\paragraph{External Links}
Hier findet man nochmal die Links zu den Webshops.

\section{Erläuterte Screenshots}

TODO
