\chapter{Konzept Web}

\section{Allgemein}

Unsere Dienstleistung konzentriert sich ähnlich der Unix Philosophie auf eine
bestimmte Tätigkeit und versucht diese möglichst gut zu realisieren. Aufgrund
dieser Tatsache ist es uns sehr wichtig den User nicht mit ,,Thematikfremden''
Themen zu nerven. Aufgrund unserer Spezialisierung werden fachfremde Kategorien
wie z.B. der Fanartikel Webshop an Drittanbieter ,,outgesourct''.
\\
\\
Um eine einfache sowie angenehme Usability zu gewährleisten konzentrieren wir
uns sehr stark darauf das \emph{Design} nach dem KISS-Prinzip so einfach wie
möglich zu halten und den Benutzer nicht mit Werbebannern oder sinnvollen
Informationen zu belästigen.

\newpage

\section{Farbklima}
Das an den Key-Visual angepasste Farbklima sieht wie folgt aus:

\begin{table}[h!]
\centering
\begin{tabular*}{\textwidth}{lll|l}

    Farbe & Name & Farbwerte & Einsatzgebiet \\
    \hline
    
\multirow{3}{*}
    {
    \begin{tikzpicture}
        \draw [line width=0.5pt,fill=a]
        (0,0) -- (1.2,0) -- (1.2,1.2) -- (0,1.2) -- cycle;
    \end{tikzpicture}
    }
    & a & HEX \#FDF9EF & Hintergrundfarbe für Contentbereich\\
    & & RGB 253, 249, 239 \\
    & &  \\
    \hline

\multirow{3}{*}
    {
    \begin{tikzpicture}
        \draw [line width=0.5pt, fill=b]
        (0,0) -- (1.2,0) -- (1.2,1.2) -- (0,1.2) -- cycle;
    \end{tikzpicture}
    }
    & b & HEX \#F4F1E5 &  Hintergrund \\ 
    & & RGB  244, 241, 229 & \\
    & &  \\
    \hline

\multirow{3}{*}
    {
    \begin{tikzpicture}
        \draw [line width=0.5pt,fill=c]
        (0,0) -- (1.2,0) -- (1.2,1.2) -- (0,1.2) -- cycle;
    \end{tikzpicture}
    }
    & c & HEX \#544738 &  Standard Schriftfarbe \\
    & & RGB 84, 71, 56 & \\
    & &  \\
    \hline

\multirow{3}{*}
    {
    \begin{tikzpicture}
        \draw [line width=0.5pt,fill=d]
        (0,0) -- (1.2,0) -- (1.2,1.2) -- (0,1.2) -- cycle;
    \end{tikzpicture}
    }
    & d & HEX \#98BF21 &   Buttonflächen \\
    & & RGB 152, 191, 33 & \\
    & &  \\
    \hline

\multirow{3}{*}
    {
    \begin{tikzpicture}
        \draw [line width=0.5pt,fill=e]
        (0,0) -- (1.2,0) -- (1.2,1.2) -- (0,1.2) -- cycle;
    \end{tikzpicture}
    }
    & e & HEX \#C2C5BE &  Footer Hintergrundfarbe \\
    & & RGB 194, 197, 190 &\\ 
    & &  \\
    \hline
   
\multirow{3}{*}
    {
    \begin{tikzpicture}
        \draw [line width=0.5pt,fill=f]
        (0,0) -- (1.2,0) -- (1.2,1.2) -- (0,1.2) -- cycle;
    \end{tikzpicture}
    }
    & f & HEX \#BBBBBB &  Tagcloud Schriftfarbe \\
    & & RGB 187, 187, 187  & \\
    & &  \\
    \hline
   
\multirow{3}{*}
    {
    \begin{tikzpicture}
        \draw [line width=0.5pt,fill=g]
        (0,0) -- (1.2,0) -- (1.2,1.2) -- (0,1.2) -- cycle;
    \end{tikzpicture}
    }
    & g & HEX \#303030 &  Tagcloud Farbe \\
    & & RGB 48, 48, 48  & \\
    & &  \\

\end{tabular*}
   \caption{Farbschema}
   \label{t_colorscheme}
\end{table}

\section{Webservice Kategorien}
Folgende Hauptkategorien wurden auf moosr umgesetzt:
\begin{itemize}
\item moosrdata
\item Developers
\item Blog
\item Webshop
\item Forum
\item About Us
\item Help
\item Login
\end{itemize}


\paragraph{moosrdata}
Hier befindet sich die der Kernpunkt unserer Dienstleistung, die Metadaten
Suchmaschine.

\paragraph{Developers}
Die Kategorie \emph{Developers} ist eine Seite mit statischem Content. Hier
werden für interessierte Entwickler alle Informationen zu unserer HTML und JSON
API erläutert.

\paragraph{Blog}
Der \emph{Blog} ist eine Seite mit wachsendem Kontent. Über diese Seite wollen
wir unsere Community auf dem laufenden halten über Newcomer aus der Szene und
Berichte über Alben, Festivals etc. verfassen.

\paragraph{Webshop}
In der Kategorie \emph{Webshop} befinden sich Seitenlinks und Beschreibungen zu
Webshops mit Tickets und Fanartikeln.

\paragraph{Forum}
Das \emph{Forum} ist der Haupttreffpunkt für die Community. Hier können sich
alle nach Lust und Laune austauschen.

\paragraph{About Us}
In der Kategorie \emph{About Us} erläutern wir dem Kunden das Spektrum unserer
Dienstleistung. Diese Seite ist zugleich unsere \emph{Hauptseite}.

\paragraph{F.A.Q.}
Der \emph{F.A.Q.} Bereich bietet interessierten Antworten auf die häufig
gestellen Fragen.

\paragraph{Login}
Hier ist der \emph{Loginbereich} der primär für \emph{Redakteure} interessant
ist die Beiträge auf der Blogseite veröffentlichen möchten.

\section{Erläuterte Screenshots}

TODO
