\chapter{Webmarketing - 20 Fragen}
\label{wettbewerb}\section{Welche besonderen Eigenschaften, Stärken,
Alleinstellungsmerkmale hat (m)eine Leistung?}

Unsere Dienstleistung bietet eine zentrale Anlaufstelle für Indie Musik
Liebhaber, Liedermacher und Künstler und kombiniert diese mit einer Metadaten Suchmaschine
die Cover-Art Suche, Songtext-Suche, Biographie-Suche und Artistphoto-Suche in
einer Dienstleistung bündelt.
\\
Anbieter mit ähnlichen Dienstleitungen: \\

\begin{itemize}
    \item http://www.allcdcovers.com
    \item http://www.albumart.org
    \item http://lyrics.wikia.com/Lyrics\_Wiki
    \item http://www.last.fm
\end{itemize}


\section{Welche besonderen Alleinstellungsmerkmale und
Vermarktungskonzepte haben Mitbewerber?}

\paragraph{Last.fm}
Der Anbieter Last.fm spezialisiert sich Hauptsächlich auf Musikstreaming.
Desweiteren bietet er eine Metadatenschnittstelle an, die Artist-Biographien,
Cover-Art und Artist-Photos liefern kann. Kompatible Clients und Webseiten können
daher diese anzeigen.

\paragraph{allcdcovers}
Dieser Anbieter hat sich auf besonders hochauflösende Cover-Art Images
spezialisiert. Außerdem bietet er zudem Bilder von der CD Rückseite und dem
Inlet an.

\paragraph{albumart}
Der Anbieter Albumart ähnelt unserem Angebot noch am ehesten. Allerdings
beschränkt er sich lediglich auf CD und DVD Cover-Art.

\paragraph{lyrics.wikia}
Dieses Angebot ist in einer Wiki-ähnlichen Struktur organisiert. Leider bietet
dieser Anbieter nur Songtexte an, allerdings verlinkt er auf andere Angebote und
integriert soziale Dienste wie Twitter und Facebook.

\section{Wie können wir unsere Leistung durch eine
    Spezialisierung abgrenzen und unsere Stärken
optimal zur Geltung bringen?}
Unsere Dienstleistung konzentriert sich auf die Bereitstellung einer
leichtgewichtigen Webschnittstelle die sowohl andere Webseiten als auch
Desktopclients nutzen können. Das Hauptaugenmerk liegt hier auf einer besonders
guten Dokumentation sowie einer einfachen Integration der Dienste in
verschiedenen externen Produkten.

Allerdings versteht sich unser Angebot als eine Art ,,Metaprovider``, der sich
auf andere Dienstleister wie beispielsweise last.fm stützt und diese bündelt.

Als Anlaufstelle für eine große Community fühlen wir uns verpflichtet diese
neben unserem Suchservice über Neuigkeiten der Indie Musik Szene sowie über
Newcomer und Festivals auf dem laufenden zu halten. 

Auf technischer Seite legen wir großen Wert auf Transparenz und die Integration
freier Software. Das Angebot soll eine schnelle Bildung einer Community
gewährleisten.


\section{Was ist die erfolgversprechendste Zielgruppe?}
Als Zielgruppe sehen wir primär Musikinteressierte Benutzer und Anbieter von
Musik-Abspielsoftware (und damit deren Nutzerbasis).
Die Zielgruppe ist prinzipiell ein junges experimentierfreudiges Publikum das
sich schnell für Neues begeistern lässt und keine Angst vor technischen Herausforderungen hat.


\section{Welche Medien nutzt die Zielgruppe?}
Primär nutzt die Zielgruppe das Internet. Andere Medien sind für unsere
Dienstleistung zu vernachlässigen.


\section{Wer sind die wichtigsten Meinungsführer?
Können wir Empfehlungs-Statements bekommen?}
Frage trifft auf unsere Dienstleistung nicht zu.


\section{Was sind die brennenden Probleme der
Zielgruppe?}
Die Zielgruppe hat das Problem Metadaten an einer zentralen Stelle im Netz
aufzufinden. Unsere Dienstleistung soll diese Angebotslücke schließen und
zusätzlich als Treffpunkt für gleichgesinnte Musikliebhaber und freischaffende
Künstler dienen.


\section{Welche Innovationsstrategie / Trojanische Pferde können wir entwickeln?
    Wie können wir einen Fuss in die Tür bekommen?
Was könnte ein zwingender Nutzen sein?}
Im Moment gibt es kein Angebot das verschiedene Metadaten maschinenlesbar 
in einem Angebot bündelt. Unser Angebot kann auf anderen Webseiten leicht durch
die leichtgewichtige Webschnitstelle eingebunden werden.

Ein zwingender Nutzen könnte sich lediglich für Musicplayer ergeben, da dort 
Client-Bibliotheken für unseren Service zur Verfügung stehen. 

\section{Welche Überraschung können wir
Meinungsführern bieten?}
Frage trifft auf unsere Dienstleistung nicht zu, da keine Meinungsführer
vorhanden.


\section{Wie können wir ein Angebot mit PR bekannt
    machen und nicht nur durch
Anzeigenwerbung?}

Durch den Einsatz von sozialen Medien wie Twitter oder Facebook kann ein
positives Meinungsbild suggeriert werden. Desweiteren kann der Service von
verschiedenen Fachseiten getestet und empfohlen werden. Aufgrund der breiten
Anwesenheit unserer Dienstleistung die sich über verschiedene soziale Netze
von Facebook bis hin zu Flicker streut sollte es problemlos möglich sein unsere
Dienstleistung einem breitem Publikum zukommen zu lassen.


\section{Welche Risiken gibt es dafür, dass die Zielgruppe
    die Leistung nicht nutzen könnte?
    Was kann potentielle Kunden eventuell
abschrecken?}
Ein mögliches Risiko wäre eine zu komplexe Entwickler unfreundliche API. Um
dieses Risiko zu minimieren wird darauf geachtet die API und Seite nach offenzen
Standards zum implementieren.

\section{Welche Kooperationsstrategie können wir
verfolgen?}
Mögliche Dienstleistungsdrittanbieter können durch gezielte Werbung auf unsere 
Dienstleistung aufmerksam gemacht werden.


\label{Markenname}\section{Können wir einen Markennamen generieren?
    Wenn das Potential gross ist, dann kann sich
dies lohnen!}
Diese Frage trifft nicht direkt zu da es sich bei moosr in erster Linie um ein
nicht kommerzielles Produkt handelt wo die Etablierung eines Markennamens
zweitrangig ist. 

\section{Ist eine Intel-Inside Strategie möglich?}
Ja, eine Intel-Inside Strategie wäre denkbar wenn verschiedene Musikplayeranbieter
moosr als Metadatensuchmaschine für nutzen. Diese Situation könnte man durch die
Implementierung von Wrapperlibraries für verschiedene Programmiersprachen
positiv beeinflussen.

Mögliche ,,Intel-Inside''-Claims für Playeranbieter:
\begin{itemize}
    \item Moosr inside.
    \item ,,My Player supports Moosrdata search!''
\end{itemize}


\section{Welches Key-Visual (Schlüsselbild) / Key-
    Theme (Schlüsselthema) können wir
verwenden? Wiedererkennung ist alles!}

Eine weiche, helle Farbgebung mit grün/braun Farbtönen die ein skandinavisches
,,feeling'' vermittelt um die zunehmend in letzter Zeit aus den skandinavischen
Ländern kommende Indie Musik Bewegung visuell zu  unterstreichen.
\\
Key-Visual und Claim siehe\ref{claim}


\section{Können wir eine Strategie entwickeln, wie wir
    mit PR und Vorträgen an die Zielgruppe oder
deren Meinungsführer direkt rankommen?}

Nein, Frage trifft nicht zu.

\section{Wie können wir einen Markenaufbau zum Nulltarif
    erreichen?
    Dazu müssen wir einen einmaligen Gattungsbegriff
für unsere Leistungen schaffen!}

Siehe\ref{Markenname} 
Wir setzen auf Mundpropaganda siehe nächste Frage.

\section{Wie können wir Mundpropaganda
    unterstützen?
    Z.B. Gewinnspiel, Spezielle Mehrwerte (z.B.
Broschüre mit Tipps,\dots)}
Bei genügend Mundpropaganda und positiver Rezession verbreitet sich der
Service automatisch über verschiedene Plattformen und Distributionen.

\label{claim}\section{Welchen Claim sollten wir verwenden?
Die treffende Message bewirkt Wunder!}

Als Claim wurde ,,moosr delivers.'' gewählt. Da dies ein kurzer, einprägsamer
Begriff ist der unsere Dienstleistung auf den Punkt bringt. Wir informieren über
Neuigkeiten und liefern Metadaten.

\section{Welches Leitbild und welche Ziele verfolgen wir
für die Ansprache der Zielgruppe?}
Frage trifft nicht auf unseren Service zu.
