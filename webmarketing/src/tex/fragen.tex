\chapter{Untersuchung (20 Fragen)}

\label{wettbewerb}
\section{Welche besonderen Eigenschaften, Stärken, Alleinstellungsmerkmale hat (m)eine Leistung?}
Unsere Dienstleistung bietet eine zentrale Anlaufstelle für Indie Musik
Liebhaber, Liedermacher und Künstler und kombiniert diese mit einer Metadaten Suchmaschine
die Cover-Art Suche, Songtext-Suche, Biographie-Suche und Artistphoto-Suche in
einer Dienstleistung bündelt.
\\
\\
Anbieter mit ähnlichen Dienstleitungen: \\

\begin{itemize}
    \item \url{http://www.allcdcovers.com}
    \item \url{http://www.albumart.org} (keine soziale Komponente)
    \item \url{http://lyrics.wikia.com/Lyrics\_Wiki}
    \item \url{http://www.last.fm}
\end{itemize}

\section{Welche besonderen Alleinstellungsmerkmale und Vermarktungskonzepte haben Mitbewerber?}
\paragraph{Last.fm}
Der Anbieter Last.fm spezialisiert sich hauptsächlich auf Musikstreaming.
Desweiteren bietet er eine Metadatenschnittstelle an, die Artist-Biographien,
Cover-Art und Artist-Photos liefern kann. Kompatible Clients und Webseiten können
daher diese anzeigen. Zudem bietet er eine soziale Komponente an, indem pro User
ein Profil vorhanden ist, auf dem einsehbar ist was dieser gerade hört bzw. mag.

\paragraph{allcdcovers}
Dieser Anbieter hat sich auf besonders hochauflösende Cover-Art Images
spezialisiert. Außerdem bietet er zudem Bilder von der CD Rückseite und dem
Inlet an. Er ist vollkommen Community-basiert.

\paragraph{albumart}
Der Anbieter Albumart ähnelt unserem Suchangebot noch am ehesten. Allerdings
beschränkt er sich lediglich auf CD und DVD Cover-Art. Er bietet zudem eine API.

\paragraph{lyrics.wikia}
Dieses Angebot ist in einer Wiki-ähnlichen Struktur organisiert. Leider bietet
dieser Anbieter nur Songtexte an. Allerdings verlinkt er auf andere Angebote und
integriert soziale Dienste wie Twitter und Facebook.
\\
\\
Wir sollten erwähnen, dass es noch eine große Anzahl weiterer Webseiten gibt.
Allerdings haben wir die oberen exemplarisch als Vergleich ausgewählt.

\section{Wie können wir unsere Leistung durch eine Spezialisierung abgrenzen und unsere Stärken optimal zur Geltung bringen?}
Unsere Dienstleistung konzentriert sich auf die Bereitstellung einer
leichtgewichtigen Webschnittstelle die sowohl andere Webseiten als auch
Desktopclients nutzen können. Das Hauptaugenmerk liegt hier auf einer besonders
guten Dokumentation sowie einer einfachen Integration der Dienste in
verschiedenen externen Produkten.
\\
\\
Allerdings versteht sich unser Angebot als eine Art ,,Metaprovider``, der sich
auf andere Dienstleister wie beispielsweise last.fm stützt und diese bündelt.
\\
Als Anlaufstelle für eine große Community fühlen wir uns verpflichtet diese
neben unserem Suchdienst über Neuigkeiten jeglicher Art auf dem Laufenden zu
halten.
\\
\\
Auf technischer Seite legen wir großen Wert auf Transparenz und die Integration
freier Software. So steht der Quellcode der Anwendung offen auf Github\footnote{\url{https://www.github.com/studenkittens/flascat}} zur
Verfügung. 



\section{Was ist die erfolgversprechendste Zielgruppe?}
Als Zielgruppe sehen wir primär musikinteressierte Benutzer und Anbieter von
Musik-Abspielsoftware (und damit deren Nutzerbasis).
Die Zielgruppe ist prinzipiell ein junges experimentierfreudiges Publikum das
sich schnell für Neues begeistern lässt.

\section{Welche Medien nutzt die Zielgruppe?}
Primär nutzt die Zielgruppe das Internet. Andere Medien sind für unsere
Dienstleistung zu vernachlässigen.

\section{Wer sind die wichtigsten Meinungsführer?}
Frage trifft auf unsere Dienstleistung nicht zu, da es in dem Sinne keine
Meinungsführer gibt.

\section{Was sind die brennenden Probleme der Zielgruppe?}
Die Zielgruppe hat das Problem Metadaten an einer zentralen Stelle im Netz
aufzufinden. Unsere Dienstleistung soll diese Angebotslücke schließen und
zusätzlich als Treffpunkt für gleichgesinnte Musikliebhaber und freischaffende
Künstler dienen.
\\
\\
Wie oben bereits erwähnt, ist die Verteilung der Communities stark dezentral. Wir wollen
das nach Möglichkeit soweit wie möglich zentralisieren und bieten ein Forum 
sowie Links auf andere Seiten.


\section{Welche Trojanische Pferde können wir entwickeln?}
Im Moment gibt es kein Angebot das verschiedene Metadaten maschinenlesbar 
in einem Angebot bündelt. Unser Angebot kann auf anderen Webseiten leicht durch
die leichtgewichtige Webschnitstelle eingebunden werden.
\\
\\
Ein zwingender Nutzen würde sich für Musicplayer ergeben, da dort 
Client-Bibliotheken für unseren Service zur Verfügung stehen.

\section{Welche Überraschung können wir Meinungsführern bieten?}
Frage trifft auf unsere Dienstleistung nicht zu, da keine Meinungsführer
vorhanden.

\section{Wie können wir ein Angebot mit PR bekannt machen und nicht nur durch Anzeigenwerbung?}
Durch den Einsatz von sozialen Medien wie Twitter oder Facebook kann ein
positives Meinungsbild suggeriert werden. Desweiteren kann der Service von
verschiedenen Fachseiten getestet und empfohlen werden. Aufgrund der hohen 
Verfügbarkeit unserer Dienstleistung die sich über verschiedene soziale Netze
von Facebook bis hin zu Flickr streut sollte es problemlos möglich sein unsere
Dienstleistung einem breitem Publikum zugänglich zu machen.

\section{Welche Risiken gibt es dafür, dass die Zielgruppe die Leistung nicht nutzen könnte? Was kann potentielle Kunden eventuell abschrecken?}
Ein mögliches Risiko wäre eine zu komplexe entwickler-unfreundliche API. Um
dieses Risiko zu minimieren wird darauf geachtet die API und Seite nach offenen
Standards zum implementieren.
\\
\\
Für normale User hätte ein kompliziertes und überladenes Design eine
abschreckende Wirkung, weswegen wir uns hier auf Minimalismus konzentrieren.

\section{Welche Kooperationsstrategie können wir verfolgen?}
Da wir außer einem externen Shop und sozialen Diensten nicht auf andere Partner
zurückgreifen trifft die Frage nicht auf unser Angebot zu.

\section{Können wir einen Markennamen generieren?}
Diese Frage trifft nicht direkt zu da es sich bei moosr in erster Linie um ein
nicht kommerzielles Produkt handelt wo die Etablierung eines Markennamens
zweitrangig ist. 

\section{Ist eine Intel-Inside Strategie möglich?}
Ja, eine Intel-Inside Strategie wäre denkbar, wenn verschiedene Musikplayeranbieter
moosr als Metadatensuchmaschine nutzen. Diese Situation könnte man durch die
Implementierung von Wrapperlibraries für verschiedene Programmiersprachen
positiv beeinflussen.
\\
\\
Mögliche ,,Intel-Inside''-Claims für Playeranbieter:
\begin{itemize}
    \item \it{Moosr bringt's} - Englisch alternativ: \it{Moosr delivers}
    \item \it{Moosr inside}
\end{itemize}

\section{Welches Key-Visual / Key-Theme können wir verwenden?}
Eine weiche, helle Farbgebung mit grün/braun Farbtönen die ein skandinavisches
,,Feeling'' vermittelt um die zunehmend in letzter Zeit aus den skandinavischen
Ländern kommende Indie Musik Bewegung visuell zu  unterstreichen.
\\
\\
Weitere Details zum KeyVisual/Claim finden sich im entsprechenden Kapitel
\ref{keyvisual_claim}.

\section{Können wir eine Strategie entwickeln, wie wir mit PR und Vorträgen an die Zielgruppe oder deren Meinungsführer direkt rankommen?}
Die Frage trifft nicht auf unsere Dienstleistungen zu.

\section{Wie können wir einen Markenaufbau zum Nulltarif erreichen?}
Wir setzen auf Mundpropaganda, siehe nächste Frage.

\section{Wie können wir Mundpropaganda unterstützen?}
Bei genügend Mundpropaganda und positiver Rezension verbreitet sich die
Dienstleistung automatisch über verschiedene Plattformen sowie z.B. auch Linux-Distributionen.
\\
\\
Dies sollte bei der hohen Verfügbarkeit in sozialen Netzen und der engen Bindung
zur Community kein Problem darstellen.

\section{Welchen Claim sollten wir verwenden? Die treffende Message bewirkt Wunder!}
Als Claim wurde ,,moosr bringt's.'' gewählt. Da dies ein kurzer, einprägsamer
Begriff ist der unsere Dienstleistung auf den Punkt bringt. Wir informieren über
Neuigkeiten und liefern Metadaten.

\section{Welches Leitbild und welche Ziele verfolgen wir für die Ansprache der Zielgruppe?}
Frage trifft nicht auf unsere Dienstleistung zu.
