\chapter{Vorwort}

\section{Motivation}

Im Netz gibt es eine riesige Menge an Seiten die sich darauf spezialisiert
haben Informationen jeglicher Art zu Musikstücken, Künstlern und Ähnlichem zu
liefern.
\\
Die Reichweite geht dabei von Songtexten bis hin zu Tagging-Informationen die von
Musikabspielprogrammen ausgelesen werden kann. Es gibt Streamingdienste wie last.fm,
oder Spotify die neben der eigentlichen Musik auch entsprechende Metadaten anbieten. 
Im Falle von last.fm sogar mit einer gut nutzbaren Webschnittstelle.
\\
Allerdings sind die ganzen Informationen mehr oder minder über das ganze Web
fragmentiert. Last.fm bietet lediglich Coverart und Künstlerbilder an, 
lyrics.wikia.com nur Lyrics und ein anderer Provider nur Biographien und
Reviews.
\\
\section{Idee}
Unsere Idee ist nun eine Metadatensuche zu bauen, die möglichst viele
Metadatentypen bündelt und dabei auch über eine Web-API nutzbar ist.
\\
Um diese Idee abzurunden, wollen wir uns einen Namen in der Indie Musik Szene
schaffen und bieten neben der Suchdienstleistung zugleich einen ,,Treffpunkt'' für Indie
Musik Liebhaber, Liedermacher und freischaffende Künstler an.

\section{Anmerkung}
Wir wollen bereits jetzt darauf hinweisen dass im Blog genannte Personen und
Ereignisse zum Teil fiktiv sind. Auch das angebotene Forum und der Webshop
existiert nur auf dem Papier. Auch viele Links, wie zB auf ,,unsere'' Flickr
oder Facebook Seite existieren nicht, da dies den Rahmen der Studienarbeit
gesprengt hätte. Die meisten Blogposts sind allerdings an reale
Bands angelehnt.
