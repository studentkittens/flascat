
\chapter{Technische Implementierung}
\section{Vorwort - technische Implementierung}
Als technische Grundlage für moosr wird ein Webframework verwendet, da dieses an
die benötigten Bedürfnisse (Suchmaschine) am relativ gut angepasst werden kann.

Als Webframework haben wir uns mit Absprache mit Herrn Prof. Dr. Alois
Kastner-Maresch auf das Python Microwebframework Flask geenigt da wir das
Framework in der Vorlesung \emph{Wiederverwendungsbasierte Entwicklung von
Systemen} in einer Studienarbeit erarbeiten und das dort gewonnene Wissen (Flask
und Python) praktisch in der Vorlesung ,,Webtechnologie und Webmarketing 
mit Open Source'' umsetzen möchten.

\subsection{Flask}
Das Microwebframework benutzt \emph{Jinja} als Template Renderengine und
\emph{Werkzeugs} als WSGI Middleware. Die Pakete selbst können mit pip gezogen
werden


Da die Webpräsentation primär als Dienstleistung zu sehen ist kommen bei der
Implementierung neben dem Framework selbst hauptsächlich auf die beiden
libraries \emph{libglyr} und \emph{sqlite3} zum Einsatz.

\subsection{libglyr}
Libglyr ist eine library zum auffinden von Musikmetadaten. 
\\
\url{https://github.com/sahib/glyr}


\subsection{sqlite3}
Sqlite3 ist eine sehr weit genutzte und bekannte embedded Datenbank, welche für
unseren Einsatz optimal ist.
\\
\url{http://www.sqlite.org/}
