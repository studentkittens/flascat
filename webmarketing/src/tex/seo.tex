\chapter{Google Optimierung}

\section{Duplicate Content}
Mit mit Framework Flask lässt sich auf einfache Art und Weise \emph{duplicate
content} vermeiden. Hier zu kann technisch die \emph{url\_for()} Funktion
verwendet werden die einem die URL eines bestimmten Elements oder einer
View-Funktion zurückliefert. Im Folgenden Beispiel wird die Funktionsweise
erläutert:

\begin{verbatim}
                        @app.route('/hello')
                        def hello_func():
                            return 'Hello World'
\end{verbatim}

Die Url \url{http://www.moosr.org/hello} wir im Beispiel auf die Funktion
\emph{hello\_dunc()} gemappt. Beim Aufruf von \emph{url\_for('hello\_func')} wir
die Url \url{http://www.moosr.org/hello} zurückgeliefert. Durch den konsistenten
Zugriff über \emph{url\_for()} wird bei bei unserem Projekt
\emph{duplicate content} vermieden.



\section{Duplicate Content vermeiden}
Durch den Einsatz von Flask als Webframework haben wir volle Kontrolle über
die Festlegung einzelner URLs. 
\section{Content Seiten über Navigation erreichbar}


\section{Artikel über mehrere Seiten strecken}
\section{Festlegbare Linktexte}
\section{Einfache URLs erzeugen}


\section{Editierbare Metadaten}

