\documentclass[11pt]{scrreprt}

% default stuff 
\usepackage[utf8]{inputenc}
\usepackage{ngerman}
\linespread{1.25}

% beautiful colors like redrubin ;)
\usepackage[usenames,dvipsnames]{xcolor}





\title{Studienarbeit Webtechnologie und Webmarketing mit OpenSource Software \\
    \textcolor{WildStrawberry}{\huge{moosr}} \\
    \normalsize{A music metadata search engine.}
}

\author{Dozent: Dr. Alois Kastner-Maresch\\
Christopher Pahl, Christoph Piechula}

\date{\today}


% let the fun begin
\begin{document}
\maketitle
\tableofcontents

\chapter{Webmarketing}

\section{Welche besonderen Eigenschaften, Stärken,
Alleinstellungsmerkmale hat (m)eine Leistung?}

\section{Welche besonderen Alleinstellungsmerkmale und
Vermarktungskonzepte haben Mitbewerber?}

\section{Wie können wir unsere Leistung durch eine
Spezialisierung abgrenzen und unsere Stärken
optimal zur Geltung bringen?}

\section{Was ist die erfolgversprechendste Zielgruppe?}

\section{Welche Medien nutzt die Zielgruppe?}

\section{Wer sind die wichtigsten Meinungsführer?
Können wir Empfehlungs-Statements bekommen?}

\section{Was sind die brennenden Probleme der
Zielgruppe?}

\section{Welche Innovationsstrategie / Trojanische
Pferde können wir entwickeln?
Wie können wir einen Fuss in die Tür
bekommen?
Was könnte ein zwingender Nutzen sein?}

\section{Welche Überraschung können wir
Meinungsführern bieten?}

\section{Wie können wir ein Angebot mit PR bekannt
machen und nicht nur durch
Anzeigenwerbung?}

\section{Welche Risiken gibt es dafür, dass die Zielgruppe
die Leistung nicht nutzen könnte?
Was kann potentielle Kunden eventuell
abschrecken?}

\section{Welche Kooperationsstrategie können wir
verfolgen?}

\section{Können wir einen Markennamen generieren?
Wenn das Potential gross ist, dann kann sich
dies lohnen!}

\section{Ist eine Intel-Inside Strategie möglich?}

\section{Welches Key-Visual (Schlüsselbild) / Key-
Theme (Schlüsselthema) können wir
verwenden? Wiedererkennung ist alles!}

\section{Können wir eine Strategie entwickeln, wie wir
mit PR und Vorträgen an die Zielgruppe oder
deren Meinungsführer direkt rankommen?}

\section{Wie können wir einen Markenaufbau zum Nulltarif
erreichen?
Dazu müssen wir einen einmaligen Gattungsbegriff
für unsere Leistungen schaffen!}

\section{Wie können wir Mundpropaganda
unterstützen?
Z.B. Gewinnspiel, Spezielle Mehrwerte (z.B.
Broschüre mit Tipps, ...)}

\section{Welchen Claim sollten wir verwenden?
Die treffende Message bewirkt Wunder!}

\section{Welches Leitbild und welche Ziele verfolgen wir
für die Ansprache der Zielgruppe?}

\chapter{Google Optimierung}
\chapter{Technische Implementierung}
\chapter{Weiteres}


\end{document}
