\documentclass[11pt]{scrreprt}

% default stuff 
\usepackage[utf8]{inputenc}
\usepackage{ngerman}
\linespread{1.25}

% beautiful colors like redrubin ;)
\usepackage[usenames,dvipsnames]{xcolor}





\title{Studienarbeit Webtechnologie und Webmarketing mit OpenSource Software \\
    \textcolor{WildStrawberry}{\huge{moosr}} \\
    \normalsize{A music metadata search engine.}
}

\author{Dozent: Dr. Alois Kastner-Maresch\\
Christopher Pahl, Christoph Piechula}

\date{\today}


% let the fun begin
\begin{document}
\maketitle
\tableofcontents

\chapter{Webmarketing}

\section{Welche besonderen Eigenschaften, Stärken,
Alleinstellungsmerkmale hat (m)eine Leistung?}

Das Produkt kombiniert Cover-Art Suche, Songtext-Suche, Biographie-Suche und
Artistphoto-Suche als eine Dienstleisung.
\\
Änbieter mit ähnlichen Dienstleitungen: \\

\begin{itemize}
    \item http://www.allcdcovers.com
    \item http://www.albumart.org
    \item http://lyrics.wikia.com/Lyrics\_Wiki
    \item http://www.last.fm
\end{itemize}


\section{Welche besonderen Alleinstellungsmerkmale und
Vermarktungskonzepte haben Mitbewerber?}

\paragraph{Last.fm}
Der Anbieter Last.fm spezialisiert sich Hauptsächlich auf Musikstreaming.
Desweiteren bietet er eine Metadatenschnittstelle an, die Artistbiographien,
Coverart und Artistphotos liefern kann. Kompatible Clients und Webseiten können
daher diese anzeigen.


\paragraph{allcdcovers}
Dieser Anbieter hat sich auf besonders hochauflösende Coverart Images
spezialisiert. Außerdem bietet er zudem Bilder von der CD Rückseite und dem
Inlet an.

\paragraph{albumart}
Der Anbieter Albumart ähnelt unserem Angebot noch am ehesten. Allerdings
beschränkt er sich lediglich auf CD und DVD Coverart.


\paragraph{lyrics.wikia}
Dieses Angebot ist in einer Wiki-ähnlichen Struktur organisiert. Leider bietet
dieser Anbieter nur Songtexte an, allerdings verlinkt er auf andere Angebote und
integriert soziale Dienste wie Twitter und Facebook.


\section{Wie können wir unsere Leistung durch eine
Spezialisierung abgrenzen und unsere Stärken
optimal zur Geltung bringen?}
Unsere Dienstleistung konzentriert sich auf die Bereitstellung einer
leichtgewichtigen Webschnittstelle die sowohl andere Webseiten als auch
Desktopclients nutzen können. Das Hauptaugenmerk liegt hier auf einer besonders
guten Dokumentation sowie einer einfachen Integration der Dienste in
verschiedenen externen Produkten.

Allerdings versteht sich unser Angebot als eine Art ,,Metaprovider``, der sich
auf andere Dienstleister wie beispielsweise last.fm stützt und diese bündelt.

Auf technischer Seite legen wir großen Wert auf Transparenz und die Integration
freier Software. Das Angebot soll eine schnelle Bildung einer Community
gewährleisten.



\section{Was ist die erfolgversprechendste Zielgruppe?}
Als Zielgruppe sehen wir primär Musikinteressierte Benutzer und Anbieter von
Musik-Abspielsoftware (und damit deren Nutzerbasis).


\section{Welche Medien nutzt die Zielgruppe?}
Die Zielgruppe wird vom Medium Internet dominiert.



\section{Wer sind die wichtigsten Meinungsführer?
Können wir Empfehlungs-Statements bekommen?}
Frage trifft auf unsere Dienstleistung nicht zu.



\section{Was sind die brennenden Probleme der
Zielgruppe?}
Die Zielgruppe hat das Problem Metadaten an einer zentralen Stelle im Netz
aufzufinden. Unsere Dienstleistung soll diese Angebotslücke schließen.


Hämorhiden.

\section{Welche Innovationsstrategie / Trojanische
Pferde können wir entwickeln?
Wie können wir einen Fuss in die Tür
bekommen?
Was könnte ein zwingender Nutzen sein?}
Im Moment gibt es kein Angebot dass verschiedene Metadaten maschinenlesbar 
in einem Angebot bündelt. Unser Angebot kann auf anderen Websiten leicht durch
die leichtgewichtige Webschnitstelle eingebunden werden.


\section{Welche Überraschung können wir
Meinungsführern bieten?}


Hämorhiden.

\section{Wie können wir ein Angebot mit PR bekannt
machen und nicht nur durch
Anzeigenwerbung?}

Durch den Einsatz von sozialen Medien wie Twitter oder Facebook kann ein
positives Meinungsbild suggeriert werden.


\section{Welche Risiken gibt es dafür, dass die Zielgruppe
die Leistung nicht nutzen könnte?
Was kann potentielle Kunden eventuell
abschrecken?}

Hämorrhoiden.

\section{Welche Kooperationsstrategie können wir
verfolgen?}

..?

\section{Können wir einen Markennamen generieren?
Wenn das Potential gross ist, dann kann sich
dies lohnen!}

Ja, können wir. Wir versuchen den Markennamen ,,moosr'' zu etablieren.
Moosicr? (Da hat er tief reingegriffen.)

\section{Ist eine Intel-Inside Strategie möglich?}

MoosrDB inside.

,,My Player supports Moosicr!''

\section{Welches Key-Visual (Schlüsselbild) / Key-
Theme (Schlüsselthema) können wir
verwenden? Wiedererkennung ist alles!}

Pink.

\section{Können wir eine Strategie entwickeln, wie wir
mit PR und Vorträgen an die Zielgruppe oder
deren Meinungsführer direkt rankommen?}

Nein.

\section{Wie können wir einen Markenaufbau zum Nulltarif
erreichen?
Dazu müssen wir einen einmaligen Gattungsbegriff
für unsere Leistungen schaffen!}

Durch Nutzung sozialer Medien wie Facebook oder Twitter die ein positives 
Meinungsbild suggerieren. 

\section{Wie können wir Mundpropaganda
unterstützen?
Z.B. Gewinnspiel, Spezielle Mehrwerte (z.B.
Broschüre mit Tipps, ...)}


\section{Welchen Claim sollten wir verwenden?
Die treffende Message bewirkt Wunder!}

,,moosicr bringt's.'' (...deine Metadaten)

\section{Welches Leitbild und welche Ziele verfolgen wir
für die Ansprache der Zielgruppe?}

1) Weltherrschaft
2) Hämorrhoiden.

\chapter{Google Optimierung}

\chapter{Technische Implementierung}
\chapter{Weiteres}


\end{document}
